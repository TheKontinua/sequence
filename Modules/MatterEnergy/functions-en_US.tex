\chapter{Functions and Their Graphs}

You can think of a function as a machine: you put something into the
machine, it processes it, and out comes something else. Just as we
often use the variable $x$ to stand in for a number, we often use the
variable $f$ to stand in for a function.

For example, we might ask, ``Let the function $f$ be defined like this:

\begin{equation*}
f(x) = -5x^2 + 12x + 2
\end{equation*}

What is the value of $f(3)$?''

You would run the number 3 through ``the machine'': $-5(3^2) + 12(3) + 2 = -7$. The answer would be ``$f(3)$ is $7$''.

Some functions are not defined for every possible input. For example:

\begin{equation*}
  f(x) = \frac{1}{x}
\end{equation*}

  This is defined for any $x$ except 0. The set of values that a function can process is called its \textit{domain}.

\begin{Exercise}[title={Domain of a function}, label=function_domain]

  Let the function $f$ be given by $f(x) = \sqrt{x - 3}$.  What is its domain?

\end{Exercise}
\begin{Answer}[ref=function_domain]
  You can only take the square root of nonnegative numbers, so the
  function is only defined when $x - 3 \geq 0$.  Thus the domain is
  all real numbers greater than or equal to 3.
\end{Answer}

\section{Graphs of Functions}

If you have a function $f$, its graph is the set of pairs $(x, y)$
such that $y = f(x)$.  We usually draw a picture of this set, so we
will use the term \textit{graph} to mean the set and also the picture.

Here is the graph of the function $f(x) = -5x^2 + 12x + 2$:

\begin{tikzpicture}
    \begin{axis}[
        xmin=-1,xmax=3.5,
        ymin=-10,ymax=11,
        axis x line=middle,
        axis y line=middle,
        axis line style=<->,
        xlabel={$x$},
        ylabel={$y$},
        ]
        \addplot[no marks,sdkblue,<->] expression[domain=-0.7:3.05,samples=100]{(-5)*(x^2) + (12 * x) + 2}; 
    \end{axis}
\end{tikzpicture}

(Note that is just part of the graph: it goes infinitely in both
directions, so I could not fit the whole thing on the page.)

Here is the graph of the function $f(x) = \frac{1}{x}$:

\begin{tikzpicture}
    \begin{axis}[
        xmin=-7,xmax=7,
        ymin=-7,ymax=7,
        axis x line=middle,
        axis y line=middle,
        axis line style=<->,
        xlabel={$x$},
        ylabel={$y$},
        ]
        \addplot[no marks,sdkblue,<->] expression[domain=-6.5:-0.15,samples=100]{1/x}; 
        \addplot[no marks,sdkblue,<->] expression[domain=0.15:6.5,samples=100]{1/x}; 
    \end{axis}
\end{tikzpicture}

\section{Can this be expressed as a function?}

Note that not all sets can be expressed as graphs of functions.  For
example, here is the set of points $(x,y)$ such that $x^2 + y^2 = 9$:

\begin{tikzpicture}
    \begin{axis}[
        xmin=-3.5,xmax=3.5,
        ymin=-3.5,ymax=3.5,
        ytick={-3,-2,-1,0,1,2,3},
        axis x line=middle,
        axis y line=middle,
        axis line style=<->,
        xlabel={$x$},
        ylabel={$y$},
        ]
        \addplot[no marks,sdkblue] expression[domain=-3:3,samples=100]{sqrt(9 - x^2)}; 
        \addplot[no marks,sdkblue] expression[domain=-3:3,samples=100]{-1 * sqrt(9 - x^2)}; 
    \end{axis}
\end{tikzpicture}

This cannot be the graph of a function because what would $f(0)$ be? 3
or -3?  This set fails what we call ``the vertical line test'': If any
vertical line contains more than one point from the set, it isn't the graph
of a function.  For example, the vertical line $x = 2$ would cross
the graph twice:

\begin{tikzpicture}
    \begin{axis}[
        xmin=-3.5,xmax=3.5,
        ymin=-3.5,ymax=3.5,
        ytick={-3,-2,-1,0,1,2,3},
        axis x line=middle,
        axis y line=middle,
        axis line style=<->,
        xlabel={$x$},
        ylabel={$y$},
        ]
        \addplot[no marks,sdkblue] expression[domain=-3:3,samples=100]{sqrt(9 - x^2)}; 
        \addplot[no marks,sdkblue] expression[domain=-3:3,samples=100]{-1 * sqrt(9 - x^2)};
        \addplot [thick, dashed] coordinates {(2,-2.5)(2,2.5)};

    \end{axis}

\end{tikzpicture}



\section{Inverses}

Some functions have inverse functions. If a function $f$is a machine that turns
number $x$ into $y$, the inverse (usually denoted $f^{-1}$) is the machine that turns $y$ back
into $x$.

For example, let $f(x) = 5x + 1$. Its inverse is
$f^{-1}(x) = (x - 1)/5$. (Spot check it: $f(3) = 16$ and $f^{-1}(16) = 3$)

Does the function $f(x) = x^3$ have an inverse? Yes, $f^{-1}(x) =
\sqrt[3]{x}$. Let's plot the function (solid line) and its inverse (dashed):

\begin{tikzpicture}
    \begin{axis}[
        xmin=-3.5,xmax=3.5,
        ymin=-3.5,ymax=3.5,
        ytick={-3,-2,-1,0,1,2,3},
        axis x line=middle,
        axis y line=middle,
        axis line style=<->,
        xlabel={$x$},
        ylabel={$y$},
        ]
        \addplot[no marks,sdkblue] expression[domain=-3:3,samples=100]{x^3}; 
        \addplot[no marks,sdkblue,dashed] expression[domain=0:3,samples=100]{x^(1/3)}; 
        \addplot[no marks,sdkblue,dashed] expression[domain=-3:0,samples=100]{-1 * (-1 * x)^(1/3)}; 
    \end{axis}
\end{tikzpicture}

The inverse is the same as the function, just with its axes swapped.
This tells us how to solve for an inverse: We swap $x$ and $y$ and
solve for $y$.

For example, if you are given the function $f(x) = 5x + 1$, its graph
is all $(x,y)$ such that $y = 5x + 1$.  The graph of its inverse is
all $(x, y)$ such that $x = 5y + 1$. So you solve for $y$: $y = (x -
1)/5$.

Not every function has an inverse.  For example, $f(x) = x^2$.  Note
that $f(2) = f(-2) = 4$.  What would $f^{-1}(4)$ be? 2 or -2?  This
implies the ``horizontal line test'': If any horizontal line contains
more than one point of a function's graph, that function has no
inverse.

\begin{tikzpicture}
    \begin{axis}[
        xmin=-3.5,xmax=3.5,
        ymin=-1, ymax=6,
        ytick={-1,0,1,2,3,4,5,6},
        axis x line=middle,
        axis y line=middle,
        axis line style=<->,
        xlabel={$x$},
        ylabel={$y$},
        ]
      \addplot[no marks,sdkblue] expression[domain=-3:3,samples=100]{x^2};
      \addplot [thick, dashed] coordinates {(-3,4)(3,4)};
    \end{axis}
\end{tikzpicture}

In some problems, you need an inverse and you don't really need the
whole domain, so you trim the domain to a set you can define an
inverse on. This allow you to say things like ``If we restrict the domain to
the nonnegative numbers, the function $f(x) = x^2 - 5$ has an inverse:
$f^{-1}(x) =\sqrt{x + 5}$.

This begs the question: What is the domain of the inverse function $f^{-1}$?

If we let $X$ be the domain of $f$, we can run every member of $X$
through ``the machine'' $f$ and gather them in a set on the other
side. This set would be the \textit{image} of $f$. (Some people call
this the \textit{range} of $f$.)

What is the image of $f(x) = x^2 - 5$? It is the set of all real
numbers greater than or equal to -5. We write this

\begin{equation*}
  \{ x \in {\rm I\!R} | x \geq -5 \}
  \end{equation*}

Now we have the words we need: \textbf{The image of the function is the domain
  of the inverse function.}

In our example, we can use any number greater
than or equal to -5 as input into the inverse function.

\begin{tikzpicture}
    \begin{axis}[
        xmin=-5.5,xmax=7.5,
        ymin=-6, ymax=5,
        xtick={-3, 2},
        ytick={-5, 1},
        axis x line=middle,
        axis y line=middle,
        axis line style=<->,
        ]
      \addplot[no marks,sdkblue, ->] expression[domain=0:3,samples=100]{x^2 - 5} node[right] {$y = x^2 - 5$};
      \addplot [thick, dashed, red, ->] coordinates {(-0.05,-5)(-0.05,4.5)}
      node [draw, red, left, align=left, yshift=-0.6cm, xshift=-0.1cm] {image of $f$ \textit{or}\\ domain of $f^{-1}$};
      \addplot [thick, dashed, blue, ->] coordinates {(-0,-5.1)(6.75,-5.1)}
      node [draw, align=left, above, blue, yshift=0.1cm, xshift=-1.3cm] {domain of $f$ \textit{or}\\image of $f^{-1}$};
    \end{axis}
\end{tikzpicture}
