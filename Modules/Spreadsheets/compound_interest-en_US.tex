\chapter{Compound Interest}

When you loan money to someone, you typically charge them some sort of
interest. The most common loan of this sort is what the bank calls a
``savings account''.  Any money you put in the account is loaned to
the bank. The bank then lends it to someone else, who pays interest to
the bank. And the bank gives some of that interest to you.

However, what if you leave the interest in your account? And you start
making \textit{interest on the interest}? This is known as
\textit{compound interest}.\index{compound interest}

\section{An example with annual interest payments}

Lets say that you put \$1000 in a savings account that pays 6\%
interest every year. How much money would you have after 12 years?
Let's make a spreadsheet.

\includegraphics[width=0.4\textwidth]{StartInterest.png}

Create a new spreadsheet and edit the cells to look like this.  All
the cells in row 1 - 4 are just values: just type in what you
see.

The fifth row is all formulas:

\begin{tabular}{c | c | c}
  After year & Interest & Balance \\
  \hline 
  = A4 + 1 & = B\$1 * C4 & = C4 + B5 \\
\end{tabular}

The interest rate field should be formatted as a percentage. One thing
to know when dealing with percentages in the spreadsheet: if the field
says ``600\%'', its value is 6. 

The cells in the Interest and Balance column should be formatted as currency.

You are about to make a bunch of copies of the cells in the fifth row,
so make sure they look right.

Click on A5 and shift click on C5 to select all three cells. Drag the
lower-right corner down to fill the rows 6 - 15.

\includegraphics[width=0.5\textwidth]{CopiedCellsInterest.png}

Look at the numbers.  The first interest payment is \$60, but the last
is \$113.90. Your balance has more than doubled!

\section{Exponential Growth}

We figured this out numerically by repeatedly multiplying the balance
by the interest rate.  What if you wanted to know what the balance
would be $n$ years after investing $P_0$ dollars with an annual interest
rate of $r$? (Note that $r$ in our example would be 0.06, not 6.0.)

Each year, the balance is multiplied by $1 + r$, so after one year,
$P_0$ would become $P_0 \times (1 + r)$.  The next year you would multiply
this number by $(1 + r)$ again: $P_0 \times (1 + r) \times (1 + r)$. The
next year? $P_0 \times (1 + r) \times (1 + r) \times (1 + r)$ See the
pattern? If we define $P_n$ be this balance after $n$ years, then

$$P_n = P_0 (1+r)^n$$

Because $n$ is an exponent, we call this \textit{exponential growth}.
And there are few things as terrifying to a scientist as
the phrase ``The population is undergoing exponential growth''.\index{exponential growth}

\section{Sensitivity to interest rate}

For most people, the first surprising thing about compound interest is
how quickly your money grows after a few years.  The second thing that
is surprising is how much difference a small change in the percentage
rate makes.

Lets add another set of columns that shows what happens to your money
if you convince the bank to pay you 8\% instead of 6\%.

Copy everything from columns B and C:

\includegraphics[width=0.6\textwidth]{CopyForSecondInterest.png}

Now edit the second interest rate to be 8\%:

\includegraphics[width=0.6\textwidth]{AtBiggerInterestRate.png}

\section{Make Bar Graph}

Try to make a bar graph that shows both balances over time:

\includegraphics[width=0.9\textwidth]{InterestGraph.png}

The year column should be used as the x-axis. There are two series of
data that come from C4:C16 and E4:E16.  Tidy up the titles and legend
as much as you like.\index{spreadsheet!graphing multiple series}

Looking at the graph, you can see the balances start the same, but
balance of the account with the larger interest rate quickly pulls
away from the account with the smaller interest rate.
