\chapter{Common Polynomial Products}

In math and physics, you will run into certain kinds of polynomials
over and over again. In this chapter, I am going to cover some
patterns that you will want to start to recognize.

\section{Difference of squares}

Watch \textbf{Polynomial special products: difference of squares} from Khan Academy at \url{https://youtu.be/uNweU6I4Icw}.

If you are asked what is $(3x - 7)(3x + 7)$, you would use the
distributive property to expand that to $(3x)(3x) + (3x)(7) + (-7)(3x) + (-7)(7)$.
Two of the terms cancel each other, so this is $(3x)^2 - (7)^2$. This would simplify to $9x^2 - 49$

You will see this pattern a lot. Anytime you see $(a + b)(a - b)$, you should immediately
recognize it equals $a^2 - b^2$. (Note that the order doesn't matter: $(a - b)(a + b)$ also $a^2 - b^2$.)

Working the other way is important too: anytime you see $a^2 - b^2$, that you should recognize that
you can change that into the product $(a + b)(a - b)$.

\begin{Exercise}[title={Difference of Squares}, label=diffsquares]
  Simply the following products
  \Question{$(2x - 3)(2x + 3)$}
  \Question{$(7 + 5x^3)(7 - 5x^3)$}
  \Question{$(x - a)(x + a)$}
  \Question{$(3 - \pi)(3 + \pi)$}
  \Question{$(-4x^3 + 10)(-4x^3 - 10)$}
  \Question{$(x + \sqrt{7})(x - \sqrt{7})$}
  Factor the following polynomials:
    \Question{$x^2 - 9$}
    \Question{$49 - 16x^6$}
    \Question{$\pi^2 - 25x^8$}
    \Question{$x^2 - 5$}
\end{Exercise}
\begin{Answer}[ref=diffsquares]
  $(2x - 3)(2x + 3) = 4x^2 - 9$
  
  $(7 + 5x^3)(7 - 5x^3) = 49 - 25x^6$
  
  $(x - a)(x + a) = x^2 - a^2$
  
  $(3 - \pi)(3 + \pi) = 9 - \pi^2$
  
  $(-4x^3 + 10)(-4x^3 - 10) = 16x^6 - 100$
  
  $(x + \sqrt{7})(x - \sqrt{7}) = x^2 - 7$

  $x^2 - 9 = (x + 3)(x - 3)$

  $49 - 16x^6 = (7 + 4x^3)(7 + 4^3)$
  
  $\pi^2 - 25x^8 = (\pi + 5x^4)(\pi - 5x^4)$
  
  $x^2 - 5 = (x + \sqrt{5})(x - \sqrt{5})$

\end{Answer}

\section{Powers of binomials}

You can raise whole polynomials to exponents. For example,
\begin{multline*}
  (3x^3 + 5)^2 = (3x^3 + 5)(3x^3 + 5) \\ = 9x^6 + 15x^3 + 15x^3 + 25 = 9x^6 + 30x^3 + 25 
\end{multline*}

Generally, for any monomials $a$ and $b$, $(a + b)^2 = a^2 + 2ab + b^2$.
So, for example, $(7x^3 + \pi)^2 = 49x^6 + 14\pi x^3 + \pi^2$

\begin{Exercise}[title={Powers of binomials}, label=binomialpower]
  Simply the following
  \Question{$(x + 1)^2$}
  \Question{$(3x^5 + 5)^2$}
  \Question{$(x^3 - 1)^2$}
  \Question{$(x - \sqrt{7})^2$}
  
\end{Exercise}
\begin{Answer}[ref=binomialpower]
  $(x+1)^2 = x^2 + 2x + 1$

  $(3x^5 + 5)^2 = 9x^10 + 30x^5 + 25$

  $(x^3 - 1)^2 = x^6 - 2x^3 + 1$

  $(x - \sqrt{7})^2 = x^2 - 2x\sqrt{7} + 7$
\end{Answer}

What about $(x + 2)^3$? You can do it as two separate multiplications:
\begin{multline*}
  (x+2)^3 = (x+2)(x+2)(x+2) \\
  = (x + 2)(x^2 + 4x + 4) = x^3 + 4x^2 + 4x + 2x^2 + 8x + 8 \\
  = x^3 + 6x^2 + 12x + 8
\end{multline*}
And, in general, we can say that for any monomials $a$ and $b$, $(a + b)^3 = a^3 + 3a^2b + 3ab^2 + b^3$.

What about higher powers? $(a + b)^4$, for example? You could use the
distributive property four times, but it starts to get pretty tedious.

Here is a trick. This is known as \emph{Pascal's triangle}
\begin{equation*}
\begin{array}{c}
 1 \\
 1 \quad 1 \\
 1 \quad 2 \quad 1 \\
 1 \quad 3 \quad 3 \quad 1 \\
 1 \quad 4 \quad 6 \quad 4 \quad 1 \\
 1 \quad 5 \quad 10 \quad 10 \quad 5 \quad 1 \\
 1 \quad 6 \quad 15 \quad 20 \quad 15 \quad 6 \quad 1 \\
 1 \quad 7 \quad 21 \quad 35 \quad 35 \quad 21 \quad 7 \quad 1
\end{array}
\end{equation*}
Each entry is the sum of the two above it.

The coefficients of each term are given by the entries in Pascal's triangle:
\begin{equation*}
(a + b)^4 = 1a^4 + 4a^3b + 6a^2 b^2 + 4 a b^3 + 1 b^4   
\end{equation*}

\begin{Exercise}[title={Using Pascal's Triangle}, label=pascalbinomial]
    \Question{What is $(x + \pi)^5$?}
\end{Exercise}
\begin{Answer}[ref=pascalbinomial]
  $(x + \pi)^5 = x^5 + 5\pi x^4 + 10\pi^2 x^3 + 10 \pi^3 + x^2 + 5 \pi^2 x + \pi^5$
\end{Answer}


