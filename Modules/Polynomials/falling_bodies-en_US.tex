\chapter{Falling Bodies}

Because of gravity, if you throw a hammer straight up in the air, from
the moment it leaves your hand until it hits the ground, it is
accelerating toward the center of the earth at a constant rate.

\emph{Acceration} is the change in velocity. If the hammer leaves your
hand with a velocity of 12 meters per second upward, one second later
it will be rising, its velocity will have slowed to 2.2 meters per
second. One second after that, the hammer will be falling at a rate of
7.6 meters per second. Every second the hammer's velocity is changing by
9.8 meters per second, and that change is always toward the center of
the earth. When the hammer is going up, gravity is slowing it down by
9.8 meters per second each second.  When the hammer is coming down,
gravity is speeding it up by 9.8 meters per second each second.\index{acceleration}

Acceleration due to gravity on earth is a constant negative 9.8 meters per second per second:
\begin{equation*}
a = -9.8   
\end{equation*}
(Why is it negative? We are talking about height, which increases as
you go away from the center of the earth. Acceleration is changing the
velocity in the opposite direction.)

\section{Calculating the Velocity}

Given that the acceleration is constant, it makes sense that the
velocity is a straight line. Assuming once again that the hammer
leaves your hand at 12 meters per second, then the upwards velocity at
time $t$ is given by:
\begin{equation*}
  v = 12 - 9.8t
\end{equation*}

Note that the velocity of the hammer is being given as a function. Here is its graph:

\begin{tikzpicture}
    \begin{axis}[
        xmin=-0.25,xmax=2.75,
        ymin=-13,ymax=13,
        axis x line=middle,
        axis y line=middle,
        axis line style=<->,
        xlabel={$t$},
        ylabel={$v$},
        ]
        \addplot[no marks,sdkblue] expression[domain=0:2.25,samples=100]{x * (-9.8) + 12} node[left] {$12 - 9.8t$}; 
    \end{axis}
\end{tikzpicture}

\begin{Exercise}[title={When is the apex of flight?}, label=vapex]
  Given the hammer's velocity is given by $12 - 9.8t$, at what time (in seconds)
  does it stop rising and begin to fall?
\end{Exercise}
\begin{Answer}[ref=vapex]
  Solve for when the velocity is zero.

  $t = \frac{12}{9.8} = 1.22$ seconds after release.
\end{Answer}

At this point, we need to say something about air resistance. Gravity
is not the only force on the hammer; as it travels through the air,
the air tries to slow it down. This force is called \emph{air resistance},
and for a large, fast moving object (like an airplane) it is very big force. For a
dense object (like a hammer) moving at a slow speed (what you generate
with your hand), air resistance doesn't significantly affect acceleration.

\section{Calculating Position}

If you let go of the hammer when it is 2 meters
above the ground, the height of the hammer is given by:
\begin{equation*}
  p = -\frac{9.8}{2}t^2 + 12t + 2
\end{equation*}

Here is a graph of this function:

\begin{tikzpicture}
    \begin{axis}[
        xmin=-1.2,xmax=3.5,
        ymin=-13,ymax=13,
        axis x line=middle,
        axis y line=middle,
        axis line style=<->,
        xlabel={$t$},
        ylabel={$p$},
      ]
      \addplot[no marks,sdkblue,dashed,<-] expression [domain=-0.7:0,samples=100] {(-4.9)*(x^2) + 12 * x + 2};
      \addplot[no marks,sdkblue] expression [domain=0:2.58,samples=100] {(-4.9)*(x^2) + 12 * x + 2};
      \addplot[no marks,sdkblue,dashed,->] expression [domain=2.58:3,samples=100] {(-4.9)*(x^2) + 12 * x + 2};
    \end{axis}
\end{tikzpicture}


How did I figure this out? \textbf{The change in position between time
  $0$ and any time $t$ is equal to the area under the velocity graph
  between $x = 0$ and $x = t$.}

Let's use the velocity graph to figure out how much the position has
changed in the first second of the hammer's flight. Here's the
velocity graph with the area under the graph for the first second filled
in:

\usepgfplotslibrary{fillbetween}

\begin{tikzpicture}
    \begin{axis}[
        xmin=-0.25,xmax=2.75,
        ymin=-13,ymax=13,
        axis x line=middle,
        axis y line=middle,
        axis line style=<->,
        xlabel={$t$},
        ylabel={$v$},
      ]
      \addplot[no marks,sdkblue, name path=f] expression[domain=0:2.25,samples=100]{x * (-9.8) + 12} node[left] {$12 - 9.8t$};
      \path[name path=xaxis] (axis cs:0,0) -- (axis cs:1,0);
      \addplot[
        thick,
        color=sdkblue,
        fill=sdkblue, 
        fill opacity=0.05
    ]
    fill between[
        of=f and xaxis,
        soft clip={domain=0:1},
    ];
    \addplot[dashed,gray] coordinates {(0,12)(1,12)};
    \addplot[dashed,gray] coordinates {(1,12)(1,0)};
    \end{axis}
\end{tikzpicture}

The blue filled region is the area of the dashed rectangle minus that
empty triangle in its upper left.  The height of the rectangle is
twelve and its width is the amount of time the hammer has been in
flight ($t$).  The triangle is $t$ wide and and $9.8t$ tall. Thus, the
area of the blue region is given by $12t - \frac{1}{2}9.8 t^2$.

That's the change in position. Where was it originally? 2 meter off
the ground.  So the height is given by $p = 2 + 12t - \frac{1}{2}9.8t^2$.
We usually write terms so that the exponent decreases, so:

$$p = - \frac{1}{2}9.8t^2 + 12t + 2$$



Finding the area under the curve like this is called
\textit{integration}.  We say ``To find a function that gives the
change in position, we just integrate the velocity function.''  A lot
of the study of calculus is learning to integrate different sorts of
functions.

One important note about integration: Any time the curve drops under
the $x$-axis, the area is considered negative. (Which makes sense,
right? If the velocity is negative, the hammer's position is
decreasing.)


\begin{tikzpicture}
    \begin{axis}[
        xmin=-0.25,xmax=2.75,
        ymin=-13,ymax=13,
        axis x line=middle,
        axis y line=middle,
        axis line style=<->,
        xlabel={$t$},
        ylabel={$v$},
      ]
      \addplot[no marks,sdkblue, name path=f] expression[domain=0:2.25,samples=100]{x * (-9.8) + 12} node[left] {$12 - 9.8t$};
      \path[name path=xaxis] (axis cs:0,0) -- (axis cs:2.25,0);
      \addplot[
        thick,
        color=sdkblue,
        fill=sdkblue, 
        fill opacity=0.05
      ]
      fill between[
        of=f and xaxis,
        soft clip={domain=0:1.2245},
      ];
      \addplot[
        thick,
        color=red,
        fill=red, 
        fill opacity=0.07
      ]
      fill between[
        of=f and xaxis,
        soft clip={domain=1.2245:2.1},
      ];
    \end{axis}
\end{tikzpicture}

A natural question at this point is ``When will the hammer hit the
ground?''  That is, when does $p = 0$. The values of $t$ where $p = 0$
are known as the \textit{roots} of the quadratic function. In the next
chapter, you will get the trick for finding the roots of any quadratic
function.
