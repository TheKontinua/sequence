\chapter{Falling Bodies}

One of the interesting uses of polynomials is in calculating the
velocity and position of something that has constant acceleration. For
example, if you throw a hammer straight up, from the moment it leaves
your hand until it hits the ground, it is accelerating at a constant
rate.

\emph{Acceration} is the change in velocity. If the hammer leaves your
hand with a velocity of 12 meters per second upward, one second later
it will be rising, its velocity will have slowed to 2.2 meters per
second. One second after that, the hammer will be falling at a rate of
7.6 meters per second. Every second hammer's velocity is changing by
9.8 meters per second, and that change is always toward the center of
the earth. When the hammer is rising, gravity is slowing it down by
9.8 meters per second each second.  When the hammer is falling,
gravity is speeding it up by 9.8 meters per second each second.

Acceleration due to gravity on earth is a constant negative 9.8 meters per second per second:
\begin{equation*}
a = -9.8   
\end{equation*}
(Why is it negative? We are talking about height, which increases as
you go away from the center of the earth. Acceleration is changing the
velocity in the opposite direction.)

Given that the acceleration is constant, it makes sense that the
velocity is a straight line. Assuming once again that the hammer
leaves your hand at 12 meters per second, then the upwards velocity at
time $t$ is given by:
\begin{equation*}
  v = 12 - 9.8t
\end{equation*}

\begin{Exercise}[title={When is the apex of flight?}, label=vapex]
  Given the hammer's velocity is given by that formula, at what time (in seconds)
  does it stop rising and begin to fall?
\end{Exercise}
\begin{Answer}[ref=vapex]
  Solve for when the velocity is zero. $t = \frac{12}{9.8}$
\end{Answer}

(At this point, I should say something about air resistance. Gravity
is not the only force on the hammer; as it travels through the air,
the air tries to slow it down. This force is called \emph{air resistance},
and for a large, fast moving object (like an airplane) it is very big force. For a
dense object (like a hammer) moving at a slow speed (what you generate
with your hand), air resistance doesn't significantly affect acceleration.)

Here is where we tie in polynomials: The height of the hammer is given
by a 2nd degree polynomial. In fact, we use 2nd degree polynomials so
much that we have given them a name; A 2nd degree polynomial is known
as a \emph{quadratic}. If you let go of the hammer when it is 2 meters
above the ground, the height of the hammer is given by:
\begin{equation*}
  h = -\frac{9.8}{2}t^2 + 12t + 2
\end{equation*}

A natural question at this point is ``When will the hammer hit the
ground?''  That is, when does $h = 0$. That is, what are the roots of
this polynomial? After this chapter, we will spend a couple of
chapters on techniques for finding the roots of polynomials.

\section{Differentiating polynomials}

If you had a function that gave you the height of an object, it would
be handy to be able to figure out a function that gave you the
velocity at which it was rising or falling. The process of converting
the position function into a velocity function is known as
\emph{differentiation} or \emph{finding the derivative}.

There are a bunch of rules for finding a derivative, but
differentiating polynomials only requires three:
\begin{itemize}
\item The derivative of a sum is equal to the sum of the derivatives.
\item The derivative of a constant is zero.
\item The derivative of a nonconstant monomial $at^b$ ($a$ and $b$ are constant numbers, $t$ is time) is $\frac{a}{b}t^{b-1}$ 
\end{itemize}

So, for example, if I tell you that the height in meters of quadcopter
at second $t$ is given by $2t^3 - 5t^2 + 9t + 200$. You could tell me
that its vertical velocity is $\frac{2}{3}t^{2} - \frac{5}{2}t + 9$

\begin{Exercise}[title={Differentation of polynomials}, label=diffpoly]
  Differentiate the following polynomials
\end{Exercise}
\begin{Answer}
\end{Answer}

Now, if you know that a position is given by a polynomial, you can
differentate it to find the object's velocity at any time.

The same trick works for acceleration: Let's say you know a function
that gives an object's velocity. To find its acceleration at any time,
you take the derivative of the velocity function.
