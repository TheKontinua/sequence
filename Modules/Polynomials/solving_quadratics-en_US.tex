\chapter{Solving Quadratics}

In the last chapter, we saw that sometimes, using trial and error, you
can guess the roots of a polynomial. Sometimes, however, it isn't so
easy. In this chapter, you will learn to solve any quadratic equation.

First, if you have an equation with polynomials of degree 2 on both
sides, you can always reduce it to something that looks like
\begin{equation*}
  x^2 + bx + c = 0
\end{equation*}

For example, if you are asked to solve $4x^2 + 8x - 19 = -2x^2 - 7$
\begin{multline*}
  4x^2 + 8x - 19 = -2x^2 - 7 \\
  6x^2 + 8x -12 = 0 \\
  x^2 + \frac{4}{3}x - 2 = 0
\end{multline*}
Here, $b = \frac{4}{3}$ and $c = -2$.

$x^2 + bx + c$ is zero when
\begin{equation*}
x = -\frac{b}{2} \pm \frac{\sqrt{b^2 - 4c}}{2}  
\end{equation*}

How do we know this is correct?

For any $b$ and $c$, the graph of $x^2 + bx + c$ is a parabola
that goes up on each end. If there are two real roots, that means the
parabola crosses the x-axis. If there are no real roots, it means the
parabola never gets low enough to touch the x-axis. If there is one
real root, it means tht the parabola just touches the x-axis.
\emph{FIXME: graph examples of each here}

Let's say that there are two real roots $p$ and $q$. Also, note that $x^2 + bx + c = (x - p)(x - q)$.
From the last chapter, we know that $p + q = -b$ and $pq = c$.

The midpoint between $p$ and $q$ is their average. Thus, $p$ and $q$ are equal distance from $-\frac{b}{2}$.
\emph{FIXME: graph with midpoint shown}

So we know that $p = -\frac{b}{2} - r$ and $q = -\frac{b}{2} + r$ for some distance $r$. ($r$, like all distances, is always non-negative.)

Remember that $pq = c$, so $c = \left(-\frac{b}{2} + r \right)\left(-\frac{b}{2} - r\right)$

Does that look familiar? It is the difference of squares.  So
$c = \frac{b}{2}^2 - r^2$, so $r^2 = \frac{b^2 - 4c}{4}$. Thus $r = \frac{\sqrt{b^2 - 4c}}{2}$

Thus, $p = -\frac{b}{2} - \frac{\sqrt{b^2 - 4c}}{2}$
and $q = -\frac{b}{2} + \frac{\sqrt{b^2 - 4c}}{2}$



Note:
\begin{itemize}
\item If $b^2 > 4c$ , there are two real numbers that satisfy the equation $x^2 + bx + c = 0$.
\item If $b^2 = 4c$ , there is one solution:$-\frac{b}{2}$.
\item If $b^2 < 4c$ , there are no real numbers that satisfy the equation..
\end{itemize}

\section{Quadratic Formula}

I like the approach I just showed you. I find it easier to remember
and easier to prove than the traditional quadratic fomula, but you
should probably know the traditional quadratic formula.  If you have $ax^2 + bx + c = 0$, then
\begin{equation*}
  x = \frac{-b \pm \sqrt{b^2 - 4ac}}{2a}
\end{equation*}
