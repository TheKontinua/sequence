\chapter{Introduction to Polynomials}

Watch Khan Academy's \textbf{Polynomials intro} video at \url{https://www.khanacademy.org/math/algebra2/x2ec2f6f830c9fb89:poly-arithmetic/x2ec2f6f830c9fb89:poly-intro/v/polynomials-intro}

A \emph{monomial}\index{monomial} is the product of a number and a variable raised to a non-negative power. Here are some monomials:
\begin{multicols}{4}
  \begin{equation*}
    3 x^2
  \end{equation*}
  \begin{equation*}
    -2 x^{15}
  \end{equation*}

  \begin{equation*}
    \pi x^2
  \end{equation*}
  \begin{equation*}
    (3.33)x^{100}
  \end{equation*}

  \begin{equation*}
    7x
  \end{equation*}
  \begin{equation*}
    3
  \end{equation*}

  \begin{equation*}
    -\frac{2}{3}x^{12}
  \end{equation*}
  \begin{equation*}
    0
  \end{equation*}

  
\end{multicols}

The exponent is called the \emph{degree} of the monomial\index{monomial!degree}. Examples: $3x^{17}$
has degree 17, $-7x$ has degree 1, and $3.2$ has degree 0 (because you can think of it as $(3.2)x^0$).

The number in the product is called the \emph{coefficient}\index{monomial!coefficient}.  Example: $3x^{17}$ has a coefficient of 3, $-2x$ has a coefficient of -2, and $(3.4)x^{1000}$ has a coefficient of 3.4.

A polynomial \index{polynomial!definition of} is the sum of one or more monomials.  Here are some polynomials:
\begin{multicols}{3}
  \begin{equation*}
    4 x^2 + 9x + 3.9
  \end{equation*}
  \begin{equation*}
    -2 x^{10} + (3.4)x - 45x^{900} - 1
  \end{equation*}
  \begin{equation*}
    \pi x^2 + \pi x + \pi
  \end{equation*}
  \begin{equation*}
    3.3
  \end{equation*}
  \begin{equation*}
   7x + 2
  \end{equation*}
  \begin{equation*}
    3x^{20}
  \end{equation*}
\end{multicols}
We say that each monomial is a \emph{term} of the polynomial.

$x^{-5} + 12$ is \emph{not} a polynomial because the first term has a negative exponent.

$x^{2} - 32x^{\frac{1}{2}} + x$ is \emph{not} a polynomial because the second term has a non-integer exponent.

$\frac{x + 2}{x^2 + x + 5}$ is \emph{not} a polynomial because it is not just a sum of monomials.
\clearpage

\begin{Exercise}[title={Identifying Polynomials}, label=findpolynomials]
    Circle only the polynomials
\begin{multicols}{3}
  \begin{equation*}
    -2 x^3 + \frac{1}{2}x + 3.9
  \end{equation*}
  \begin{equation*}
    2 x^{-10} + 4x - 1
  \end{equation*}
  
  \begin{equation*}
    (4.5)x^2 + \pi x
  \end{equation*}
  \begin{equation*}
    x^{\frac{2}{3}}
  \end{equation*}
  
  \begin{equation*}
   7
  \end{equation*}
  \begin{equation*}
    3x^{20} + 2x^{19} -5 x^{18}
  \end{equation*}
\end{multicols}
\end{Exercise}

\begin{Answer}[ref=findpolynomials]
\begin{multicols}{3}
  \begin{equation*}
    \boxed{-2 x^3 + \frac{1}{2}x + 3.9}
  \end{equation*}
  \begin{equation*}
    2 x^{-10} + 4x - 1
  \end{equation*}

  \begin{equation*}
    \boxed{(4.5)x^2 + \pi x}
  \end{equation*}
  \begin{equation*}
    x^{\frac{2}{3}}
  \end{equation*}

  \begin{equation*}
   \boxed{7}
  \end{equation*}
  \begin{equation*}
    \boxed{3x^{20} + 2x^{19} -5 x^{18}}
  \end{equation*}
\end{multicols}

\end{Answer}
