\chapter{Volumes of Common Solids}

The volume of a rectangular solid is the product of its three
dimensions.  So if a block of ice is 5 cm tall, 3 cm wide and, 2 cm
deep, it's volume is $5 \times 3 \times 2 = 30$ cubic centimeters.

\tdplotsetmaincoords{80}{130} 
\begin{tikzpicture} [scale=1, tdplot_main_coords, axis/.style={->,sdkblue}, 
light vector/.style={-stealth,dashed,very thick, black}, 
vector/.style={-stealth,black,very thick}, 
vector guide/.style={dashed,sdkblue}]

%standard tikz coordinate definition using x, y, z coords
\coordinate (O) at (0,0,0);

%draw axes
\draw[axis] (0,0,0) -- (3,0,0) node[anchor=north east]{$x$};
\draw[axis] (0,0,0) -- (0,4,0) node[anchor=north west]{$y$};
\draw[axis] (0,0,0) -- (0,0,5.2) node[anchor=south]{$z$};

%draw a vector from O to P
\draw[thick,black] (0,0,0) -- (0,0,5);
\draw[thick,black] (0,3,5) -- (0,0,5);
\draw[thick,black] (2,0,5) -- (0,0,5);

\draw[thick,black] (0,0,0) -- (0,3,0);
\draw[thick,black] (0,3,5) -- (0,3,0) node[midway, right]{$5$ cm};
\draw[thick,black] (2,3,0) -- (0,3,0) node[midway, below]{$2$ cm};;

\draw[thick,black] (0,0,0) -- (2,0,0);
\draw[thick,black] (2,0,5) -- (2,0,0);
\draw[thick,black] (2,3,0) -- (2,0,0) node[midway,below]{$3$ cm};

\draw[thick,black] (2,3,5) -- (2,0,5);
\draw[thick,black] (2,3,5) -- (0,3,5);
\draw[thick,black] (2,3,5) -- (2,3,0);

\end{tikzpicture}


A
cubic centimeter is the same as a milliliter. A milliliter of ice
weighs about 0.92 grams.  So the block of ice would have a mass of $30
\times 0.92 = 27.6$ grams. \index{volume ! rectangular solid}

A sphere with a radius of $r$ has a volume of \index{volume ! sphere}

$$v = \frac{4}{3} \pi r^3$$

(For completeness, the surface area of that sphere would be

$$a  = 4 \pi r^2$$

Note that a circle of radius $r$ is one quarter of ths: $\pi r^2$.)

\begin{Exercise}[title={Flying Sphere}, label=flying_sphere]

An iron sphere is traveling at 5 m/s. (It is not spinning.)  The
sphere has a radius of 1.5 m.  Iron has a density of 7,800 kg per
cubic meter.  How much kinetic energy does the sphere have?
\end{Exercise}
\begin{Answer}[ref=flying_sphere]
  The volume of the sphere (in cubic meters) is

  $$\frac{4}{3}\pi (1.5)^3 = 4.5 \pi \approx 14.14$$

  The mass (in kg) is $14.14 \times 7800 \approx 110,269$

  The kinetic energy (in joules) is

  $$k = \frac{110269 \times 5^2}{2} = 1,378,373$$

  About 1.4 million joules.
\end{Answer}

\section{Cylinders}

The base and the top of a right cylinder are identical circles. The
circles are on parallel planes.  The sides are perpendicular to those
planes.

\tdplotsetmaincoords{75}{0} 
\begin{tikzpicture} [scale=4, tdplot_main_coords, axis/.style={->,sdkblue}]
\draw[dashed, sdkblue] (0.5,0,0) -- (0.5,0,0.7) node[midway, right]{$h$};
\draw[dashed, sdkblue] (0.5,0,0) -- (1,0,0) node[midway, below]{$r$};
\draw[black] (0.5, 0, 0) circle (0.5);
\draw[black] (0.5, 0, 0.7) circle (0.5);
\draw[black] (0,0,0) -- (0,0,0.7);
\draw[black] (1,0,0) -- (1,0,0.7);
\draw[black] (0.5,0,0) circle (0.02);
\draw[black] (0.5,0,0.7) circle (0.02);
\end{tikzpicture}

The volume of the a right cylinder of radius $r$ and height
$h$ is given by:

$$v = \pi r^2 h$$

That is, it is the area of the base times the height.

What if the base and top are identical, but the sides aren't
perpendicular to the base? This is called \newterm{oblique cylinder}.

\tdplotsetmaincoords{75}{0} 
\begin{tikzpicture} [scale=4, tdplot_main_coords, axis/.style={->,sdkblue}]
\draw[dashed, sdkblue] (0.7,0,0) -- (0.7,0,0.7) node[midway, right]{$h$};
\draw[dashed, sdkblue] (0.5,0,0) -- (1,0,0) node[midway, below]{$r$};
\draw[black] (0.5, 0, 0) circle (0.5);
\draw[black] (0.7, 0, 0.7) circle (0.5);
\draw[black] (0,0,0) -- (0.2,0,0.7);
\draw[black] (1,0,0) -- (1.2,0,0.7);
\draw[black] (0.5,0,0) circle (0.02);
\draw[black] (0.7,0,0.7) circle (0.02);
\end{tikzpicture}

The volume is still the height times the area of the base.  Note,
however, that the height is measured perpendicular to the bottom and
top.

\begin{Exercise}[title={Tablet}, label=tablet]

  A drug company has to create a tablet with volume of 90 cubic millimeters.

  The tablet will be a cylinder with half spheres on each end.  The radius will be 2mm.

  How long do they need to make the tablet to be?

  \vspace{2mm}
  
\tdplotsetmaincoords{90}{20} 
\begin{tikzpicture} [scale=6.5, tdplot_main_coords, axis/.style={->,sdkblue}]
\draw[dashed, sdkblue] (-0.2,0,0) -- (0,0,0);
\draw[dashed, sdkblue] (0,0,0) -- (0,0,0.2) node[midway, right]{2 mm};
\draw[dashed, sdkblue] (0, 0, 0) [x={(0,0,1)}] circle (0.2);
\draw[dashed, sdkblue] (0.5, 0, 0) [x={(0,0,1)}] circle (0.2);
\draw[dashed, sdkblue] (0.5,0,0) -- (0.7,0,0);
\draw[black] (0, 0, -0.2) [y={(0,0,-1)}] arc (90:270:0.2);
\draw[black] (0.5, 0, 0.2) [y={(0,0,-1)}] arc (-90:90:0.2);
\draw[black] (0,0,0.2) -- (0.5,0,0.2);
\draw[black] (0,0,-0.2) -- (0.5,0,-0.2);
\draw[dashed, sdkblue] (-0.2, 0, 0) -- (-0.2, 0, -0.3);
\draw[dashed, sdkblue] (0.7, 0, 0) -- (0.7, 0, -0.3);
\draw[dashed, sdkblue] (-0.2, 0, -0.3) -- (0.7, 0, -0.3) node[midway, below]{?};
\end{tikzpicture}
  

\end{Exercise}
\begin{Answer}[ref=tablet]
  In your mind, you can dissemble the tablet into a sphere (made up of
  the two ends) and a cylinder (between the two ends)
  
  The volume of the sphere (in cubic millimeters) is

  $$\frac{4}{3}\pi (2)^3 =\frac{32}{3}\pi \approx 33.5$$

  Thus the cylinder part has to be $90 - 33.5 = 56.5$ cubic mm. The
  cylinder part has a radius of 2 mm. If the length of the cylinder
  part is $x$, then

  $$\pi 2^2 x = 56.5$$

  Thus $x = \frac{56.5}{4 \pi} \approx 4.5$ mm.

  The cylinder part of the table needs to be 4.5mm.  Thus the entire tablet is 8.5mm long.
  
\end{Answer}
