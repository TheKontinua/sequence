\chapter{Vectors and Matrices}

THIS CHAPTER IS UNDER MAJOR CONSTRUCTION. IGNORE IT FOR NOW


You've come a long way in your problem solving journey. You've used algebra to solve simple equations like $7x + 10 = 24$ and quadratic equations like $4x^{2} + 9x + 31 = 0$. Algebra works with scalars, regular numbers that represent values. You also learned about vectors, quantities that represent both a magnitude and direction. Recall the parachute jumper in Workbook 5. To figure out the force, and in turn how fast the parachutist is traveling, you needed to know the direction and speed of the wind, the direction and speed of the plane, the wind resistance, and gravity. A scalar can represent speed, but to take direction into account, you need to use vectors. That's where linear algebra comes in.    

Linear algebra is a specialized form of algebra that represents and manipulate linearly related variables. It uses vectors and matrices to represent variables. A vector is a one-dimensional array of numbers. A matrix is a multidimensional array of numbers. In this chapter, a matrix will be two dimensions. You can think of a matrix as a collection of vectors. 


\section{Vector-Matrix Multiplication}
Let's take a look at the general form of vector-matrix multiplication. Given a matrix $A$ of size $m \times n$ and a vector $x$ of size $n \times 1$, the product $Ax$ is a new vector of size $m \times 1$. 

You compute the $i$-th component of the product vector $Av$ by taking the dot product of the $i$-th row of $A$ and the vector $v$:

\begin{equation*}
(Av)_i = \sum_{j=1}^n a_{i,j}x_j
\end{equation*}

where $a_{i,j}$ is the element in the $i$-th row and $j$-th column of $A$, and $v_j$ is the $j$-th element of $v$.

This is the general form of a matrix and a vector, written to show the specific components of each:


 $$A = \begin{bmatrix}
 a_{1,1} & a_{1,2}  & a_{1,3} & ... & a_{1,n}  \\
 a_{2,1} & a_{2,2}  & a_{2,3} & ... & a_{2,n}  \\
 ... \\
 a_{m,1} & a_{m,2}  & a_{m,3} & ... & a_{m,n}  
\end{bmatrix}$$

$$v = \begin{bmatrix}
 v_{1}  \\
 v_{2} \\
 v_{3} \\
 ... \\
 v_{m} 
\end{bmatrix}$$

 $$Av =\begin{bmatrix}
 v_{1}*a_{1,1} +v_{2}*a_{1,2}  +v_{3}*a_{1,3} +... +v_{m}*a_{1,n}  \\
 v_{1}*a_{2,1} +v_{2}*a_{2,2}  +v_{3}*a_{2,3} +... +v_{m}*a_{2,n}  \\
 ... \\
 v_{1}*a_{m,1} +v_{2}*a_{m,2}  +v_{3}*a_{m,3} +... +v_{m}*a_{m,n}  
\end{bmatrix}$$

Let's look at a specific example.

$$A = \begin{bmatrix}
 2  & 4 & 6  \\
 3  & 5 & 7  \\
 1  & 2 & 3  \\
 8  & 6 & 2 
\end{bmatrix}$$

$$v = \begin{bmatrix}
 -2  \\
 1 \\
 3 
\end{bmatrix}$$

Solution:
$$= \begin{bmatrix}
-2*2+1*4+3*6\\
-2*3+1*5+3*7\\
 -2*1+1*2+3*3\\
-2*8+1*6+3*2
\end{bmatrix}$$

$$= \begin{bmatrix}
18 \\
20\\
9\\
-4 
\end{bmatrix}$$
$$= (18,20,9,-4)$$

\begin{Exercise}[title={Vector Matrix Multiplication}, label=vector-matrix-multiply01]
Multiply the array $A$ with the vector $v$. Compute this by hand, and make sure to show your computations. 
$$A = \begin{bmatrix}
1 & -2  & 3 & 5  \\
-4  & 2  & 7 & 1 \\
3  & 3  & -9 & 1
\end{bmatrix}$$
	$$v = 
	\begin{bmatrix}
		2 \\
 		2 \\
 		6 \\
 		-1
	\end{bmatrix}$$
\end{Exercise}
\begin{Answer}[ref=vector-matrix-multiply01]
$$Av = (11 37 -43)$$
\end{Answer}

\subsection{Vector-Matrix Multiplication in Python}
Most real-world problems use very large matrices where it becomes impractical to perform calculations by hand. As long as you understand how matrix-vector multiplication is done, you'll be equipped to use a computing language, like Python, to do the calculations for you. 

Create a file called \filename{vectors\_matrices.py} and enter this code:

\begin{Verbatim}
// import the python module that supports matrices
import numpy as np

// create an array
a = np.array([[5, 1, 3, -2], 
              [1, -1, 8, 4], 
              [6, 2, 1, 3]])

// create a vector 
b = np.array([1, 2, 3, -8])

//calculate the dot product
print(a.dot(b))
\end{Verbatim}

When you run it, you should see:
\begin{Verbatim}
[16, 6, 8]
\end{Verbatim}

\section{Where to Learn More}
Watch this video from Khan Academy about matrix-vector products: \url{https://rb.gy/frga5}

