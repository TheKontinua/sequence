\chapter{Rational Functions}

We have discussed addition, subtraction, and multiplication of polynomials. What about division?

A quotient of polynomials is called a rational function. When the polynomials are factored and the stars align, we can simplify the rational expression to a single polynomial, just like we might reduce a fraction to lowest terms.

\textbf{Example} 
\begin{equation} \label{eq1}
\begin{split}
\frac{(x + 1)(x + 5)}{x + 5} & = (x + 1) * \frac{x+5}{x+5} \\
& = x + 1
\end{split}
\end{equation}

What if the polynomials are not factored? Factor them first.

\textbf{Example} 
\[ \frac{x^2 + 6x + 5}{x + 5} = \frac{(x + 1)(x + 5)}{x + 5} \]
and simplify as in the previous example.

Now, let us consider a rational expression which can be simplified to a single polynomial - but in the denominator.

\textbf{Example}
\begin{equation} \label{eq1}
\begin{split}
\frac{x + 5}{x^2 + 6x + 5} & = \frac{x + 5}{(x + 1)(x + 5)} \\
& = \frac{1}{x+1} * \frac{x+5}{x+5} \\
& = \frac{1}{x+1} \\
\end{split}
\end{equation}

Consider this expression as a function: \( f(x) = \frac{1}{x+1} \). As you might have guessed, this is called a rational function. We did not bother looking at the result of the previous example as a function, because we already know that function type: it is a line with slope \( 1 \) and y-intercept \( 1 \). But this rational function is another animal entirely. Let us examine our first rational function with a familiar concept: the y-intercept.

y-intercept: \( f(0) = \frac{1}{0 + 1} = \frac{1}{1} = 1 \). The graph contains the point \( (0, 1) \).

Does \( f \) have an x-intercept? That would be an \( x \)-value where \( f(x) = 0 \). But a fraction equals \( 0 \) only when its numerator equals \( 0 \); since the numerator of this expression is always \( 1 \), \( f \) has no x-intercept. 

Knowing the \( y \)-intercept, and that there is no \( x \)-intercept, is a comforting start. But things get weird when we consider a concept that has previously seemed quite simple: domain. Recall that the domain of a function is the set of all values which can be used as inputs. In this case, the domain includes all real numbers, with one exception. The number \( -1 \) is not a valid input because \( f(-1) = \frac{1}{-1+1} = \frac{1}{0} \), which is undefined. So, we say that the domain is all real numbers except \( -1 \). This means the graph contains a point corresponding to every \( x \)-value except \( -1 \).

There is no point at \( x = -1 \), but there is a point at every other \( x \)-value, such as, say, \( -1.1 \), or \( -0.99999 \). So what is happening near \( x = -1 \)?

\begin{center}
\begin{tabular}{ |c|c|c|c|c|c|c| } 
 \hline
 x & -1.1 & -1.01 & -1.001 & -0.999 & -0.99 & -0.9 \\ 
 \hline
 f(x) & -10 & -100 & -1000 & 1000 & 100 & 10 \\ 
 \hline
\end{tabular}
\end{center}

The function is going haywire: as we choose \( x \)-values closer and closer to \( -1 \), the resulting function values are larger and larger in magnitude. Also, they are negative on one side, but positive on the other. So how does a graph go from \( y \)-values of \( -10 \), to \( -100 \), to \( -1000 \), all in a space of less than \( 0.1 \) on the \( x \)-axis? And then all of a sudden to big positive numbers on the other side of \( x = -1 \)? All without ever crossing the \( x \)-axis (since there is no \( x \)-intercept)? Let us look at the graph.

\begin{figure}[htbp]
  \centering
  \begin{tikzpicture}
    \begin{axis}[
      axis lines = middle,
      xlabel = \(x\),
      ylabel = \(f(x)\),
      restrict y to domain = -10:10,
      samples = 100,
    ]
    \addplot [blue, smooth] {1/(x + 1)};
    \end{axis}
  \end{tikzpicture}
  \caption{Graph of \( f(x) = \frac{1}{x+1} \)}
\end{figure}

We can see the \( y \)-intercept we found above. We can also see that the graph has no \( x \)-intercept, as expected. The phenomenon occurring at \( x = -1 \) is called a vertical asymptote. One other interesting feature of this graph is how it hugs the \( x \)-axis toward the left and right edges of the window. This makes the line \( y = 0 \) (the \( x \)-axis) a horizontal asymptote for this function. We can see why this is happening numerically by considering what happens for \( x \)-values far from \( 0 \). In this function, the result is a fraction with a numerator of \( 1 \) and a denominator that is large in size: a fraction that is close to \( 0 \).

\begin{center}
\begin{tabular}{ |c|c|c|c|c|c|c| } 
 \hline
 x & -1000 & -100 & -10 & 10 & 100 & 1000 \\ 
 \hline
 f(x) & -0.001 & -0.01 & -0.1 & 0.1 & 0.01 & 0.001 \\ 
 \hline
\end{tabular}
\end{center}

Let us examine another rational function. Begin by factoring to see if the function can be simplified.
\[ g(x) = \frac{x^2 - 3x + 2}{x^2 - 4x + 3} = \frac{(x - 1)(x - 2)}{(x - 1)(x - 3)} \]
Consider the domain of \( g \) before continuing. Which values of \( x \) are valid inputs? Since substituting \( x = 1 \) or \( x = 3 \) would result in division by \( 0 \), these are not valid inputs. The domain of \( g \) is all real numbers except \( 1 \) and \( 3 \).

Now, for any \( x \)-value except \( 1 \), \( \frac{x-1}{x-1} = 1 \). This means that, for all \( x \)-values but \(1\), we can cancel those factors, leaving \( g(x) = \frac{x-2}{x-3} \). (We will talk more about what is happening at \( x = 1 \) in a moment.)

This function has both \( x \)- and \( y \)-intercepts:
\( y \)-intercept: \( g(0) = \frac{0-2}{0-3} = \frac{2}{3} \). The graph contains the point \( (0, \frac{2}{3}) \).
\( x \)-intercept: \( g(x) = 0 \) where the numerator equals \( 0 \) and the denominator does not equal \( 0 \). Since \( x - 2 = 0 \) when \( x = 2 \), the \( x \)-intercept is \( 2 \) and the graph contains the point \( (2, 0) \).

The graph of \( g \) has a vertical asymptote at any \( x \)-value where substitution would result in dividing a nonzero number by zero. Thus, \( g \) has a vertical asymptote at \( x = 3 \).

Does \( g \) have a horizontal asymptote? Let us see what happens when we substitute \( x \)-values far from \( 0 \).

\begin{center}
\begin{tabular}{ |c|c|c|c|c|c|c| } 
 \hline
 x & -1000 & -100 & -10 & 10 & 100 & 1000 \\ 
 \hline
 g(x) & 0.999 & 0.990 & 0.923 & 1.143 & 1.010 & 1.001 \\ 
 \hline
\end{tabular}
\end{center}

As we move further away from the \( y \)-axis, the \( y \)-values become closer to \( 1 \). The horizontal asymptote describes the end behavior of the function, or what the graph looks like far from the \( y \)-axis. In this case, if we ignore the portion close to the \( y \)-axis, the graph begins to look like the line \( y = 1 \), making this the horizontal asymptote of \( g \). 

So, what is happening at \( x = 1 \)? The value is not in the domain of the function, but there is no vertical asymptote there. That is because substituting any other value for \( x \), even values very close to \( 1 \), into \( \frac{(x - 1)(x - 2)}{(x - 1)(x - 3)} \) gives the exact same number as substituting into \( \frac{x-2}{x-3} \). So, there is a hole in the graph at \( x = 1 \), but nothing strange is happening on either side of \( 1 \). (Depending on the graphing software, the hole may not be visible.)

\begin{figure}[htbp]
  \centering
  \begin{tikzpicture}
    \begin{axis}[
      axis lines = middle,
      xlabel = \(x\),
      ylabel = \(g(x)\),
      restrict y to domain = -10:10,
      samples = 100,
    ]
    \addplot [red, smooth] {(x - 2)/(x - 3)};
    \end{axis}
  \end{tikzpicture}
  \caption{Graph of \( g(x) = \frac{x^2 - 3x + 2}{x^2 - 4x + 3} \)}
\end{figure}

In those examples, common factors cancel, leaving one polynomial. Of course, there is no guarantee that any two polynomials will have common factors, or even be factorable at all. Now, we consider an example which cannot be simplified. We will focus on just the asymptotes here.
\[ h(x) = \frac{x^2}{x - 1} \]
We see that the \( x \)-value \( 1 \) gives division of a non-zero number by zero, giving a vertical asymptote at \( x = 1 \). How about a horizontal asymptote? We examine values of \( h \) for values of \( x \) far from \( 0 \).

\begin{center}
\begin{tabular}{ |c|c|c|c|c|c|c| } 
 \hline
 x & -1000 & -100 & -10 & 10 & 100 & 1000 \\ 
 \hline
 h(x) & -999 & -99 & -9 & 11 & 101 & 1001 \\ 
 \hline
\end{tabular}
\end{center}

Rather than seeing function values leveling off as in the previous examples, we see function values that grow in size along with \( x \). The function \( h \) has no horizontal asymptote. Let us examine the graph:

\begin{figure}[htbp]
  \centering
  \begin{tikzpicture}
    \begin{axis}[
      axis lines = middle,
      xlabel = \(x\),
      ylabel = \(h(x)\),
      restrict y to domain = -10:10,
      samples = 100,
    ]
    \addplot [green, smooth] {x^2/(x - 1)};
    \end{axis}
  \end{tikzpicture}
  \caption{Graph of \( h(x) = \frac{x^2}{x - 1} \)}
\end{figure}

This function exhibits a different type of end behavior: that of a line with slope \( 1 \). To see that, cover up the portion of the graph near the \( y \)-axis and focus on the left and right. The rather dull and time-consuming technique of polynomial long division can be used to rewrite the function as a quotient and a remainder. Feel free to watch the Khan Academy video on the topic, but let us instead use our knowledge of factoring techniques and a clever little trick.

\begin{equation} \label{eq1}
\begin{split}
h(x) & = \frac{x^2}{x - 1} \\
& = \frac{x^2 - 1 + 1}{x - 1} \\ 
& = \frac{x^2 - 1}{x - 1} + \frac{1}{x - 1} \\
& = \frac{(x - 1)(x + 1)}{x - 1} + \frac{1}{x - 1} \\
& = x + 1 + \frac{1}{x - 1}
\end{split}
\end{equation}

We obtain a quotient of \( x + 1 \) and a remainder of \( 1 \). It is the quotient which determines the end behavior of the graph. Why? Substituting \( x \)-values far from zero makes the remainder term very small, since it becomes a fraction with a large denominator but a numerator of only \( 1 \). So for \( x \)-values far from zero, the \( y \)-value is \( x \) plus \( 1 \) plus a very small number, so small that we can justifiably ignore it. This means that far from the \( y \)-axis, the function acts like the quotient: the line \( y = x + 1 \). We call this line an oblique asymptote. See below how the graph of \( h(x) \) hugs that line.

\begin{figure}[htbp]
  \centering
  \begin{tikzpicture}
    \begin{axis}[
      axis lines = middle,
      xlabel = \(x\),
      ylabel = \(y\),
      restrict y to domain = -10:10,
      samples = 100,
    ]
    \addplot [green, smooth] {x^2/(x - 1)};
    \addplot [black, dashed] {x + 1};    
    \end{axis}
  \end{tikzpicture}
  \caption{Graph of \( h(x) = \frac{x^2}{x - 1} \) and its oblique asymptote \( y = x + 1 \)}
\end{figure}

We have seen lines act as end behaviors. Are there other possibilities? Sure! Here is an example with parabolic end behavior. 
\[ k(x) = \frac{x^3}{x - 2} \]
We use our add-subtract trick to reveal the quotient, which describes the end behavior.

\begin{equation} \label{eq1}
\begin{split}
h(x) & = \frac{x^3}{x - 2} \\
& = \frac{x^3 - 8 + 8}{x - 2} \\ 
& = \frac{x^3 - 8}{x - 2} + \frac{8}{x - 2} \\
& = \frac{(x - 2)(x^2 + 2x + 4)}{x - 2} + \frac{8}{x - 2} \\
& = x^2 + 2x + 4 + \frac{8}{x - 2}
\end{split}
\end{equation}

The quotient, \( x2 + 2x + 4 \), should describe the end behavior. We confirm by graphing both \( k \) and the quotient - the parabolic asymptote.

\begin{figure}[htbp]
  \centering
  \begin{tikzpicture}
    \begin{axis}[
      axis lines = middle,
      xlabel = \(x\),
      ylabel = \(y\),
%      restrict y to domain = -10:10,
      samples = 100,
      xmin = -5, xmax = 7, ymin = -21, ymax = 51,
    ]
    \addplot [purple, smooth] {x^3/(x - 2)};
    \addplot [black, dashed] {x^2 + 2*x + 4};    
    \end{axis}
  \end{tikzpicture}
  \caption{Graph of \( k(x) = \frac{x^3}{x - 2} \) and its parabolic asymptote \( y = x^2 + 2x + 4 \)}
\end{figure}
