\chapter{Derivatives}

In calculus, the derivative of a function represents the rate at which
the function is changing at a particular point. It is a fundamental
concept that has vast applications in various fields, including
physics.\index{derivative}

\section{Definition}

The derivative of a function $f(x)$ at a point $x$ is defined as the limit:

\begin{equation}
f'(x) = \lim_{{h \to 0}} \frac{f(x+h) - f(x)}{h}
\end{equation}

provided this limit exists. In words, the derivative of $f$ at $x$ is
the limit of the rate of change of $f$ at $x$ as the change in $x$
approaches zero.

\section{Applications in Mathematics}
\subsection{l'Hospital's Rule}
Consider the function $h(x) = \frac{\ln{x}}{x-1}$ and suppose we are interested in the behavior of $h(x)$ around $x=1$. If we apply the Quotient Rule, we get an indeterminate result: $$\lim_{x \to 1}\frac{\ln{x}}{x-1} = \frac{0}{0}$$ Looking at the graph of $h(x)$, we can guess that $\lim_{x \to 1}\frac{\ln{x}}{x-1} = 1$. 

\begin{tikzpicture}
\begin{axis}
    [clip=true,
    xmin=0, xmax=4,
    ymin=0, ymax=4,
    axis lines=left
    ]
    \addplot[blue, domain=0.01:4, samples=100]{ln(x)/(x-1)};
    \addplot[samples=50, black, dashed] coordinates{(1, 0)(1,1)};
    \addplot[samples=50, black, dashed] coordinates{(0, 1)(1,1)};
    \addplot[mark=*, fill=white, draw=blue] coordinates{(1, 1)};
\end{axis}
\end{tikzpicture}

Let's examine the numerator and denominator separately: we'll define $f(x)=\ln{x}$ and $g(x) = x-1$. 

\begin{tikzpicture}
    \begin{axis}
    [clip=false,
    xmin=0, xmax=3,
    ymin=-1, ymax=2,
    axis lines = center]
    \addplot[blue, samples=100, domain=1/e:3]{ln(x)}
    node[right, pos=1]{$f(x)$};
    \addplot[red, samples=100, domain=0:3]{x-1}
    node[right, pos=1]{$g(x)$};
    \end{axis}
\end{tikzpicture}

If we zoom in very far around $x=1$, the graphs begin to look linear:

\begin{tikzpicture}
    \begin{axis}
    [clip=false,
    xmin=0.9, xmax=1.1,
    ymin=-0.1, ymax=0.1,
    axis lines = center]
    \addplot[blue, samples=100, domain=0.9:1.1]{ln(x)}
    node[right, pos=0.9]{$f(x)$};
    \addplot[red, samples=100, domain=0.9:1.1]{x-1}
    node[right, pos=1]{$g(x)$};
    \end{axis}
\end{tikzpicture}

We can approximate these graphs as linear functions with slopes $m_1$ and $m_2$, so that the blue curve is approximated as $y=m_1(x-1)$ and the red curve is approximated as $y=m_2(x-1)$. The ratio of the functions would then be $$\frac{m_1(x-1)}{m_2(x-1)}=\frac{m_1}{m_2}$$ which is the same as the ratio of the derivatives of our linear approximations. This suggests l'Hospital's rule, that the limit of a ratio is the same as the limit of the ratio of the derivatives for certain indeterminate forms: $$\lim_{x\to a}\frac{f(x)}{g(x}=\lim_{x\to a}\frac{f'(x)}{g'(x)}$$.

Let's apply l'Hospital's rule to our limit of $h(x)$:
$$\lim_{x\to 1}\frac{\ln{x}}{x-1}=\lim_{x \to 1}\frac{\frac{d}{dx}\ln{x}}{\frac{d}{dx}(x-1)}=\lim_{x \to 1}\frac{\frac{1}{x}}{1}=1$$

Notice our result with l'Hospital's rule matches our guess based on the graph of $h(x) = \frac{\ln{x}}{x-1}$. 

L'Hospital's rule also applies to the indeterminate result $\frac{\pm \infty}{\pm \infty}$. For a limit of the form $\lim_{x\to a}\frac{f(x)}{g(x)}$, l'Hospital's rule applies if:
\begin{enumerate}
    \item the original limit is of the indeterminate form $\frac{0}{0}$ or $\frac{\pm \infty}{\pm \infty}$
    \item $f$ and $g$ are differentiable on an interval containing $a$ (but possibly not differentiable at $a$)
    \item $g'(x) \neq 0$ on said interval
\end{enumerate}

\section{Applications in Physics}

In physics, derivatives play a vital role in describing how quantities
change with respect to one another.

\subsection{Velocity and Acceleration}

In kinematics, the derivative of the position function with respect to
time gives the velocity function, and further taking the derivative of
the velocity function gives the acceleration function. For example, if
$s(t)$ represents the position of an object at time $t$, then the
velocity $v(t)$ and acceleration $a(t)$ are given by:

\begin{equation}
v(t) = \frac{ds}{dt} \quad \text{and} \quad a(t) = \frac{dv}{dt} = \frac{d^2s}{dt^2}
\end{equation}

\subsection{Force and Momentum}

In mechanics, the derivative of the momentum of an object with respect
to time gives the net force acting on the object, as stated by
Newton's second law of motion:

\begin{equation}
F = \frac{dp}{dt}
\end{equation}

where $F$ is the force, $p$ is the momentum, and $t$ is the time.

