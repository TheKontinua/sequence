\chapter{Methods of Integration}

\section{u-substitution}
Sometimes a function's antiderivative isn't obvious. Take this 
integral for example: $$\int 4x \sqrt{1 + 2x^2}\, dx$$\\ We can solve 
this integral using \textit{u-substitution}. Recall from implicit 
differentiation that if $u = f(x)$, then we can also say $du = 
f'(x) dx$. Let's set $u$ so that it is equal to the statement under 
the square root sign: $$u = 1 + 2x^2$$\\ Taking the derivative of both 
sides, we see that $$du = (4x) dx$$ How does this help us evaluate 
the integral? First, let's rearrange the integrand a bit: $$\int 4x 
\sqrt{1 + 2x^2}\,dx = \int \sqrt{1 + 2x^2} 4x\,dx$$\\ We can substitute 
$u = 1 + 2x^2$ and $du = 4x dx$ to get: $$= \int \sqrt{u}\,du$$\\ That 
is a much nicer integral! We can evaluate this integral using the 
Power Rule: $$\int \sqrt{u}\,du = \frac{2}{3}u^{3/2}$$\\ We can now 
substitute $u = 1 + 2x^2$ back into our solution to yield: $$= 
\frac{2}{3}(1 + 2x^2)^{3/2}$$\\ Feel free to double-check this answer 
by taking the derivative using the Chain Rule. You should get the 
original integrand, $4x \sqrt{1 + 2x^2}$, back. 

As you may have guessed, u-substitution is a method to help us "undo" 
the Chain Rule. Recall that the Chain Rule states: $$\frac{d}{dx}
f(g(x)) = f'(g(x))g'(x)$$\\ If we integrate both sides we see that: 
$$f(g(x)) = \int f'(g(x))g'(x)\,dx$$ Which leads us to the formal 
definition of the u-substitution method:\\

If u = g(x) is a differentiable function whose range is an interval 
$I$ and $f$ is continuous on $I$, then $\int f(g(x))g'(x)\,dx = \int 
f(u)\,du$

\section{Partial Fractions}

\section{Integration by Parts}

\section{Practice}
\begin{Exercise}[label=int_meth1]
Using the substitution $u = x^2 - 3$, re-write $\int_{-1}^4 x(x^2 - 3)
^5\.dx$ in terms of $u$.
\end{Exercise}

\begin{Answer}[ref=int_meth1]
If $u = x^2 - 3$, then $du = 2x dx$ and $x(x^2 - 3)^5 dx = \frac{1}{2}
u^5 du$. When $x = -1$, $u = -2$ and when $x = 4$, $u = 13$. Putting 
it all together, we find an equivalent integral is $\frac{1}{2}\int
_{-2}^{13} u^5\,du$. 
\end{Answer}

\begin{Exercise}[label = int_meth2]
	Evaluate $\int_0^1 \frac{5x + 8}{x^2 + 3x + 2}\,dx$ without a 
	calculator. 
\end{Exercise}

\begin{Answer}[ref=int_meth2]
	We cannot use u-substitution because $\frac{d}{dx}(x^2 + 3x + 2) \neq 
	n(5x + 8)$. We will use partial fractions to simplify the integrand. 
	Settig up: $\frac{5x + 8}{(x + 1)(x + 2)} = \frac{A}{x + 1} + 
	\frac{B}{x + 2}$. Rearranging, we find $5x + 8 = A(x + 2) + B(x + 1)$. 
	Letting $x = -2$, we find that $B = 2$. And taking $x = -1$, we find 
	$A = 3$. Therefore, $\int_0^1 \frac{5x + 8}{x^2 + 3x + 2}\,dx = \int
	_0^1 \frac{3}{x + 1}\,dx + \int_0^1 \frac{2}{x + 2}\,dx$. Evaluating 
	the integrals, we get $3\ln{(x + 1)}|_0^1 + 2\ln{(x + 2)}|_0^1 = 3(
	\ln{2} - \ln{1}) + 2(\ln{3} - \ln{2}) = 3\ln{2} + 2\ln{\frac{3}{2}} 
	= \ln{8} + \ln{\frac{9}{4}} = \ln{\frac{8 \cdot 9}{4}} = \ln{18}$. 
\end{Answer}

\begin{Exercise}[label=int_meth3]
Let $f$ be a function such that $\int f(x) \sin{x}\,dx = -f(x)\cos{x} 
+ \int 4x^3 \sin{x}\,dx$. Give a possible expression for $f(x)$. 
\end{Exercise}

\begin{Answer}[ref=int_meth3]
This question takes the form of integration by parts. That is, $\int 
f(x)g'(x)\,dx = f(x)g(x) - \int g(x)f'(x)\,dx$. If we let $g(x) = 
-\cos{x}$, then $g'(x) =\sin{x}$. The structure of the equation 
implies that $f'(x) = 4x^3$ and therefore that $f$ could be $f(x) = 
x^4$. 
\end{Answer}

\begin{Exercise}[label=int_meth4]
Evaluate $\int_1^{\infty}xe^{-x^2}\,dx$. 
\end{Exercise}

\begin{Answer}[ref=int_meth4]
Letting $u = -x^2$, then $du = -2x dx$ and $x dx = \frac{-1}{2}du$. 
Substituting $u$ and $du$ into the integral, we have $\int_{x = 1}^{x 
= \infty} \frac{-1}{2}e^u\,du$, which equals $\frac{-1}{2}e^u = 
\frac{-1}{2}e^{-x^2}|_1^{\infty}$. Evaluating the statement, we get 
$\frac{-1}{2}(e^{-\infty} - e^{-1}) = \frac{-1}{2}(0-\frac{1}{e}) = 
\frac{1}{2e}$
\end{Answer}