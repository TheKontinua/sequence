\chapter{Implicit Differentiation}

Implicit differentiation is a technique in calculus for finding the derivative of a relation defined implicitly, that is, a relation between variables $x$ and $y$ that is not explicitly solved for one variable in terms of the other. 

\section{Implicit Differentiation Procedure}

Consider an equation that defines a relationship between $x$ and $y$:

\[
F(x, y) = 0
\]

To find the derivative of $y$ with respect to $x$, we differentiate both sides of this equation with respect to $x$, treating $y$ as an implicit function of $x$:

\[
\frac{d}{dx} F(x, y) = \frac{d}{dx} 0
\]

Applying the chain rule during the differentiation on the left side of the equation gives:

\[
\frac{\partial F}{\partial x} + \frac{\partial F}{\partial y} \frac{dy}{dx} = 0
\]

Finally, we solve for $\frac{dy}{dx}$ to find the derivative of $y$ with respect to $x$:

\[
\frac{dy}{dx} = -\frac{\frac{\partial F}{\partial x}}{\frac{\partial F}{\partial y}}
\]

This result is obtained using the implicit differentiation method.

\section{Example}

Consider the equation of a circle with radius $r$:

\[
x^2 + y^2 = r^2
\]

Differentiating both sides with respect to $x$, we get:

\[
2x + 2y \frac{dy}{dx} = 0
\]

Solving for $\frac{dy}{dx}$ gives:

\[
\frac{dy}{dx} = -\frac{x}{y}
\]

which is the slope of the tangent line to the circle at any point $(x, y)$.
