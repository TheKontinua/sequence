\chapter{Riemann Sums}

\begin{Exercise}[label=rsum1]
	\begin{center}
		\begin{tabular}{c|c|c|c|c}
			t (hours) & 4 & 7 & 12 & 15\\
			R(t) (L/hr) & 6.5 & 6.2 & 5.9 & 5.6\\
		\end{tabular}
	\end{center}
	A tank contains 50 liters of water after 4 hours of filling. Water is being added to the tank at rate $R(t)$. The value of $R(t)$ at select times is shown in the table. Using a right Riemann sum, estimate the amount of water in the tank after 15 hours of filling. 
\end{Exercise}

\begin{Answer}[ref=rsum1]
	The volume of water will be the amount of water at 4 hours (50 liters) plus the area under the graph of $R(t)$ from $t=4$ to $t=15$. We will estimate this area with a right Riemann sum. The approximate volume added from $t=4$ to $t=7$ is $(7-4)*(6.2) = 18.6$ liters. The approximate volume added from $t=7$ to $t=12$ is $(12-7)*(5.9)=29.5$ liters. The approximate volume added from $t=12$ to $t=15$ is $(15-12)*(5.6) = 16.8$ liters. Therefore, the approximate total volume of water in the tank at $t=15$ is $50 + 18.6 + 29.5 + 16.8 = 114.9$ liters. 
\end{Answer}