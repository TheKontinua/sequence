\chapter{Optimization}

Optimization is a branch of mathematics that involves finding the best solution from all feasible solutions. In the field of operations research, optimization plays a crucial role. Whether it is minimizing costs, maximizing profits, or reducing the time taken to perform a task, optimization techniques are employed to make decisions effectively and efficiently.

\section{Optimization Problems}
An optimization problem consists of maximizing or minimizing a real function by systematically choosing the values of real or integer variables from within an allowed set. This function is known as the objective function.

A standard form of an optimization problem is:

\begin{equation*}
\begin{aligned}
& \underset{x}{\text{minimize}}
& & f(x) \
& \text{subject to}
& & g_i(x) \leq 0, ; i = 1, \ldots, m \
&
& & h_j(x) = 0, ; j = 1, \ldots, p
\end{aligned}
\end{equation*}

where
\begin{itemize}
\item $f(x)$ is the objective function,
\item $g_i(x) \leq 0$ are the inequality constraints,
\item $h_j(x) = 0$ are the equality constraints.
\end{itemize}

\section{Types of Optimization Problems}
There are different types of optimization problems, including but not limited to:

\begin{itemize}
\item \textbf{Linear Programming:} The objective function and the constraints are all linear.
\item \textbf{Integer Programming:} The solution space is restricted to integer values.
\item \textbf{Nonlinear Programming:} The objective function and/or the constraints are nonlinear.
\item \textbf{Stochastic Programming:} The objective function and/or constraints involve random variables.
\end{itemize}

These problems are solved using different techniques and algorithms, many of which are a subject of active research.

\section{Applications}
Optimization techniques have a wide variety of applications in many fields such as economics, engineering, transportation, and scheduling problems.

