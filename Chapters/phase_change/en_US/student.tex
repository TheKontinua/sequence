\chapter{Phases of Matter}

You have experienced $H_2O$ in three phases of matter:
\begin{itemize}
\item Ice is $H_2O$ in the solid phase.  At standard pressure,  when the temperature of $H_2O$ is below $0^\circ$ C,  it is a solid.  
\item Water is $H_2O$ in the liquid phase.  At standard pressure, when the temperature of $H_2O$ is between $0^\circ$ C and  $100^\circ$ C,  it is a liquid.
\item Water vapor (or steam) is the gas phase.  At standard pressure,  when the temperature of $H_2)$ is above $100^\circ$ C,  it is a gas.
\end{itemize}

Let's look at some of the properties of the three phases:

\begin{tabular}{p{5cm}|p{5cm}|p{5cm}}
Gas & Liquid & Solid \\
\hline
Assumes the volume and shape of its container & 
Assumes the shape, but not the volume, of its container &
Retains its shape and volume \\
\hline
Compressible & Not compressible & Not compressible \\
\end{tabular}

\section{Thinking Microscopically About Phase}

As mentioned in an early chapter,  there are intermolecular forces that attract molecules to each
other.   A pair of molecules will have very strong intermolecular forces or very weak intermolecular forces
depending on what atoms they are made of.

For example,  two helium molecules are very weakly attracted to each other due to weak intermolecular forces.   Two molecules of $NaCl$ (table salt) will experience very strong intermolecular attraction.

In a gas,  the molecules have lots of room to roam and lots of kinetic energy: The intermolecular attraction has very little effect.

In a liquid,  the molecules are sticking close together,  but are still moving around,  sort of like bees in a hive.

In a solid, the molecules are not changing their configuration, and the kinetic energy they have is just expressed as vibrations within that configuration.   You can imagine them like eggs in a carton just vibrating.

As you would expect, molecules with strong intermolecular attraction require more kinetic energy to change phases.  For example,   helium is a liquid below $-269\circ$ C.  $NaCl$, on the other hand, is a liquid between $801^\circ$ and $1,413^\circ$ C.  

The temperatures I just gave you are at standard pressure (101.325 kPa or 1 atm).  Pressure also has a role in phase change:  In low pressure environments,  it is much easier for the molecules to make the jump to being a gas.

For example,  if you climb a mountain until the atmospheric pressure is 70 kPa,  your water will boil at about $90^\circ$ C.  

If you rise in a balloon until the atmospheric pressure is 500 Pa,  if your water is colder than $-2^\circ$ C,  it will be ice.  If it is warmer it will vaporize.    There is no liquid water at 500 Pa!

For any molecule,  we could observe its phase at a wide range of temperatures and pressures.  This would let us create a phase diagram.  Here is the phase diagram for $H_2O$:

\includegraphics[width=\textwidth]{waterphase_edit.png}

(FIXME: This diagram needs to be recreated prettier.)

\section{Phase Changes and Energy}

The molar heat capacity of ice is about 37.7 J/mol-K.  That is it takes about 37.7 Joules of energy to raise the temperature of one mole of ice by one degree kelvin.

The molar heat capacity of liquid water is about 75.4 J/mol-K.  For water vapor, it is about 36.6 J/mol-K.

Imagine you have mole of ice at $173^\circ$ K  and you are going gradually add kinetic energy into it until you have steam at $473^\circ$ K.  You might guess (wrongly) that the temperature vs. energy applied would look like this:

\includegraphics[width=0.8\linewidth]{energynaive.png}

However,  once molecules are nestled into their solid state (like eggs in cartons),  it take extra energy to make them move like a liquid.  How much more energy?  For water,  it is 6.01 Joules per mole.

Similarly,  the transition from liquid to gas takes energy.  At standard pressure,  converting a mole of liquid water to vapor requires 40. 7 Joules per mole.  So the graph would actually look like this:

\includegraphics[width=0.8\linewidth]{energysoph.png}

Note that just as melting and vaporizing require energy.   Going the other way (freezing and condensing, respectively) give off energy.    Thus, we can store energy using the phase change.

\section{How a Rice Cooker Works}

\includegraphics[width=0.9\linewidth]{riceCookerPhase.png}




